\chapter{Physical observables}
\section{Introduction}
My equilibrium simulation program currently supports the calculation of the following physical observables:
  selfconsistent superconducting gap;
  charge and spin supercurrents;
  singlet/triplet decomposition of these currents;
  induced magnetization;
  and density of states.
Now that we are moving to nonequilibrium simulations, these equations need to be generalized.

First of all, the selfconsistency equation for the superconducting gap needs to be replaced with a version that depends on the nonequilibrium distribution.
The supercurrent equations need to be replaced with nonequilbrium versions, and in addition, we wish to calculate resistive currents now.
The induced magnetization needs to be supplemented with nonequilibrium charge and spin accumulation.
Furthermore, in nonequilibrium systems we might want to calculate heat currents and heat accumulation too.
Finally, since we're already rewriting a lot of the code for physical observables anyway, we might as well implement support for the spin-polarization of the density of states as well, which could be useful in the theoretical analysis.

This leaves one unaddressed point in the list above: the singlet/triplet decomposition of the charge current.
I believe that such a decomposition should still be possible in a general nonequilibrium setting, however Morten's current derivation assumes an equilibrium distribution function, and a nonequilibrium generalization would probably be more complex.
I therefore won't generalize this equation to nonequilibrium for now, but we can get back to this in the future if it turns out that we'll need it at some point.

Using a Pauli-decomposition of the distribution function -- as in the paper with Tom -- is very efficient for parametrizing the nonequilibrium Usadel equation itself, since it permits a straight-forward way to separate the equilibrium and nonequilibrium quantities in the equation.
However, when it comes to calculating physical observables, most quantities turn out to be easier to express directly in terms of either the propagator~$\U{G}^K$ or matrix current~$\U{I}^K$, instead of performing an explicit Pauli-decomposition of the equations.
I therefore focused on this strategy in the derivations that follow.

TODO: Divide all currents and accumulations by 2.


\clearpage
\section{Matrix current}
The full $8\times8$ matrix current is given by the expression:
\begin{align}
  \UV{I} 
  &= 
  \UV{G} \nabla \UV{G}
  \\ &= 
  \begin{pmatrix} 
    \U{G}^R  & \U{G}^K \\[1ex] 0 & \U{G}^A
  \end{pmatrix}
  \begin{pmatrix} 
    \nabla\U{G}^R  & \nabla\U{G}^K \\[1ex] 0 & \nabla\U{G}^A
  \end{pmatrix}
  \\ &= 
  \begin{pmatrix} 
    \U{G}^R\nabla\U{G}^R  & \U{G}^R\nabla\U{G}^K + \U{G}^K\nabla\U{G}^A \\[1ex] 0 & \U{G}^A\nabla\U{G}^A
  \end{pmatrix}
\end{align}
So the Keldysh component of the matrix current is:
\begin{align}
  \U{I}^K = \U{G}^R\nabla\U{G}^K + \U{G}^K\nabla\U{G}^A
\end{align}
Let us now substitute in $\U{G}^K = \U{G}^R \U{H} - \U{H} \U{G}^A$:
\begin{align*}
  \U{I}^K 
  &= 
  \U{G}^R\nabla(\U{G}^R \U{H} - \U{H} \U{G}^A)+ (\U{G}^R \U{H} - \U{H} \U{G}^A) \nabla\U{G}^A 
  \\ &=
  \U{G}^R(\nabla\U{G}^R) \U{H} + \U{G}^R\U{G}^R (\nabla\U{H}) - \U{G}^R (\nabla\U{H}) \U{G}^A 
  \\ &
  - \U{G}^R \U{H} (\nabla\U{G}^A) + \U{G}^R \U{H}(\nabla\U{G}^A) - \U{H} \U{G}^A (\nabla\U{G}^A)
  \\ &=
  \big[ (\U{G}^R \nabla \U{G}^R) \U{H} - \U{H} (\U{G}^A \nabla \U{G}^A) \big]
  \\ &\,+
  \big[ (\nabla\U{H}) - \U{G}^R (\nabla\U{H}) \U{G}^A \big]
\end{align*}
Let us give these two bracketed contributions to $\U{I}^K$ separate names:
\begin{align}
  \U{S}\, &\equiv (\U{G}^R \nabla \U{G}^R) \U{H} - \U{H} (\U{G}^A \nabla \U{G}^A),  \\
  \U{R}\, &\equiv (\nabla\U{H}) - \U{G}^R (\nabla\U{H}) \U{G}^A.
\end{align}
The first contribution $\U{S}$ is the \emph{supercurrent contribution} to the matrix current, which is independent of any potential gradients $\nabla\U{H}$, and can be nonzero even in equilibrium.
This is the matrix current equivalent to what Chandrasekhar and Tom called $\BC{Q}\BC{H}$.
As for the second contribution $\U{R}$, this is the \emph{resistive contribution} to the matrix current, which is only present when we have a potential gradient $\nabla\U{H}$.
This is what Chandrasekhar and Tom called $\BC{M}\nabla\BC{H}$.

\clearpage
\section{Energy symmetries}
In this section, I point out some new symmetries I noticed in the propagators, which will be useful to simplify the results derived in the following sections.
First, let us write out the structure of the propagators and distribution:
\begin{align}
  \U{G}^R &= 
  \begin{pmatrix}
    +\U{g}^R  & +\U{f}^R  \\[1ex]
   -\TU{f}^R & -\TU{g}^R
  \end{pmatrix} \\
  \U{G}^A &= 
  \begin{pmatrix}
    +\U{g}^A  & +\U{f}^A  \\[1ex]
   -\TU{f}^A & -\TU{g}^A
  \end{pmatrix} \\
  \U{G}^K &= 
  \begin{pmatrix}
    +\U{g}^K  & +\U{f}^K  \\[1ex]
   +\TU{f}^K & +\TU{g}^K
  \end{pmatrix} \\
  \U{H} \;&= \;
  \begin{pmatrix}
    \PM\U{h}  & \PM0\PM  \\[1ex]
    \PM0      & -\TU{h}\PM
  \end{pmatrix}
\end{align}
By explicit calculations, one can show that multiplying by $\N{1}$ from the left is equivalent to permuting rows, while multiplying by $\N{1}$ from the right permutes columns.
If we do both, we see from the structures above that:
\begin{align}
  \N{1} \U{G}^R \N{1}   &= -\TU{G}^R \\
  \N{1} \U{G}^A \N{1}   &= -\TU{G}^A \\
  \N{1} \U{G}^K \N{1}   &= +\TU{G}^K \\
  \N{1} \,\U{H}\, \N{1} &= -\TU{H}
\end{align}
Since tilde-conjugating is per definition a combination of complex conjugation $i \rightarrow -i$ and energy reversal $\epsilon \rightarrow -\epsilon$, this means that we get some simple identities relating the positive- and negative-energy propagators:
\begin{align}
  \U{G}^R(-\epsilon) &= -\N{1}\U{G}^{R*}(+\epsilon)\,\N{1} \\
  \U{G}^A(-\epsilon) &= -\N{1}\U{G}^{A*}(+\epsilon)\,\N{1} \\
  \U{G}^K(-\epsilon) &= +\N{1}\U{G}^{K*}(+\epsilon)\,\N{1} \\
  \U{H}(-\epsilon)   &= -\N{1}\U{H}^{*}(+\epsilon) \,\N{1}
\end{align}
If we substitute this into the matrix currents from the previous section:
\begin{align}
  \U{I}^K(-\epsilon) &= -\N{1} \U{I}^{K*}(+\epsilon)\N{1}, \\
  \U{S}(-\epsilon)\, &= -\N{1} \U{S}^*(+\epsilon)\, \N{1}, \\
  \U{R}(-\epsilon)\, &= -\N{1} \U{R}^*(+\epsilon)\, \N{1}.
\end{align}


\clearpage
\section{Charge and spin accumulation}
The charge density $\mu_e = e\mu_0$ and spin density $\B{\mu}_s = (\hbar/2)(\mu_1, \mu_2, \mu_3)$ can be found by performing the four integrals ($n=0,\ldots,3$):
\begin{equation}
  \mu_n = -\frac{1}{8}N_F  \int_{-\infty}^{+\infty} \d\epsilon \tr\big[ \S{n} \U{G}^K \big].
\end{equation}
We can then use the previously derived identity $\U{G}^K(-\epsilon) = \N{1} \U{G}^{K*}(+\epsilon) \,\N{1}$ to rewrite this as an integral over only positive energies:
\begin{equation}
  \mu_n = -\frac{1}{8}N_F  \int_{0}^{\infty} \d\epsilon \tr\big[ \S{n} (\U{G}^K + \N{1} \U{G}^{K*} \N{1})\big].
\end{equation}
We then use the cyclic trace rule to get rid of the superfluous Pauli matrices:
\begin{equation}
  \mu_n = -\frac{1}{8}N_F  \int_{0}^{\infty} \d\epsilon \tr\big[ \S{n} (\U{G}^K + \U{G}^{K*})\big].
\end{equation}
Recognizing this as simply the real part of the propagator, we get:
\begin{equation}
  \mu_n = -\frac{1}{4}N_F  \int_{0}^{\infty} \d\epsilon \re\tr\big[ \S{n} \U{G}^K \big].
\end{equation}
In order to separate the induced magnetization from the nonequilibrium spin accumulation, the propagator $\U{G}^K$ in the equations above may be replaced with difference from equilibrium $\U{G}^K-\U{G}^K_\text{\,eq}$ as in e.g. the Silaev paper.

Note: one detail I glossed over in the derivation above, is that $\S{n}$ should really be a $4\times4$ matrix $\text{diag}(\S{n},\S{n}^*)$.
In this case, $\text{diag}(\S{2}, \S{2}^*) = \N{3}\S{2}$ is different from the other spin components, as it gets one sign flip under complex conjugation, and another sign flip when sandwiched by $\N{1}$ matrices because $\N{1}\N{3}\N{1}=-\N{3}$.
The final result above should however be the same.

\clearpage
\section{Heat and spin-heat accumulation}
The heat density $\kappa_{e} = \kappa_0$ and spin-heat density $\B{\kappa}_{s} = (\kappa_1, \kappa_2, \kappa_3)$ may be calculated from the four integrals ($n=0,\ldots,3$):
\begin{equation}
  \kappa_n = -\frac{1}{8}N_F  \int_{-\infty}^{+\infty} \d\epsilon \tr\big[ \epsilon\N{3}\S{n} \U{G}^K \big].
\end{equation}
This may again be rewritten as an integral over positive energies using $\U{G}^K(-\epsilon) = \N{1} \U{G}^{K*}(+\epsilon) \N{1}$, but we get a sign flip because the prefactor $\epsilon \rightarrow -\epsilon$:
\begin{equation}
  \kappa_n = -\frac{1}{8}N_F  \int_{0}^{\infty} \d\epsilon \tr\big[ \epsilon\N{3}\S{n} (\U{G}^K - \N{1} \U{G}^{K*} \N{1}) \big].
\end{equation}
We then use the cyclic rule and the fact that $\N{1}\N{3}\N{1} = -\N{3}$:
\begin{equation}
  \kappa_n = -\frac{1}{8}N_F  \int_{0}^{\infty} \d\epsilon \tr\big[ \epsilon\N{3}\S{n} (\U{G}^K + \U{G}^{K*}) \big].
\end{equation}
We again recognize the integrand as the real part of the propagator:
\begin{equation}
  \kappa_n = -\frac{1}{4}N_F  \int_{0}^{\infty} \d\epsilon \re\tr\big[ \epsilon\N{3}\S{n} \U{G}^K \big].
\end{equation}
As before, the equilibrium and nonequilibrium contributions may be separated by replacing $\U{G}^K \rightarrow \U{G}^K - \U{G}^K_\text{eq}$ in the equation above.



\clearpage
\section{Charge and spin currents}
The charge current $I_e = eI_0$ and spin current $\B{I}_s = (\hbar/2)(I_1,I_2,I_3)$ can be calculated from the four integrals ($n=0,\ldots,3$):
\begin{equation}
  I_n &= \frac{1}{8}N_F D \int_{-\infty}^{+\infty} \d\epsilon \tr\big[ \N{3} \S{n} \U{I}^K \big]
\end{equation}
Using $\U{I}^K(-\epsilon) = -\N{1} \U{I}^{K*}(+\epsilon) \N{1}$, the cyclic rule, and $\N{1}\N{3}\N{1} = -\N{3}$ like before:
\begin{equation} 
  I_n &= \frac{1}{8}N_F D \int_{0}^{\infty} \d\epsilon \tr\big[ \N{3} \S{n} (\U{I}^K + \U{I}^{K*}) \big]
\end{equation}
Recognizing this as the real part of the matrix current, we get:
\begin{equation} 
  I_n &= \frac{1}{4}N_F D \int_{0}^{\infty} \d\epsilon \re\tr\big[ \N{3} \S{n} \U{I}^K \big]
\end{equation}
By replacing the total matrix current $\U{I}^K$ by the supercurrent $\U{S}$ and resistive current $\U{R}$, respectively, we then obtain the corresponding contributions to the charge and spin currents:
\begin{align} 
  I_{n,s} &= \frac{1}{4}N_F D \int_{0}^{\infty} \d\epsilon \re\tr\big[ \N{3} \S{n} \U{S} \big]\\
  I_{n,r} &= \frac{1}{4}N_F D \int_{0}^{\infty} \d\epsilon \re\tr\big[ \N{3} \S{n} \U{R} \big]
\end{align}


\clearpage
\section{Heat and spin-heat currents}
The heat current $J_e = J_0$ and spin-heat current $\B{J}_s = (J_1, J_2, J_3)$ can be calculated from the four integrals ($n=0,\ldots,3$):
\begin{equation}
  J_n &= \frac{1}{8}N_F D \int_{-\infty}^{+\infty} \d\epsilon \tr\big[ \epsilon \S{n} \U{I}^K \big]
\end{equation}
Using $\U{I}^K(-\epsilon) = -\N{1} \U{I}^{K*}(+\epsilon) \N{1}$, the cyclic rule, and $\epsilon \rightarrow -\epsilon$ like before:
\begin{equation}
  J_n &= \frac{1}{8}N_F D \int_{0}^{\infty} \d\epsilon \tr\big[ \epsilon \S{n} (\U{I}^K + \U{I}^{K*}) \big]
\end{equation}
We again recognize this as the real part of the current:
\begin{equation}
  J_n &= \frac{1}{4}N_F D \int_{0}^{\infty} \d\epsilon \re\tr\big[ \epsilon \S{n} \U{I}^K \big]
\end{equation}
To split the result into contributions due to supercurrents and resistive currents, we replace $\U{I}^K$ by either $\U{S}$ or $\U{R}$ in the equation above:
\begin{align}
  J_{n,r} &= \frac{1}{4}N_F D \int_{0}^{\infty} \d\epsilon \re\tr\big[ \epsilon \S{n} \U{R} \big] \\
  J_{n,s} &= \frac{1}{4}N_F D \int_{0}^{\infty} \d\epsilon \re\tr\big[ \epsilon \S{n} \U{S} \big]
\end{align}



\clearpage
\section{Superconducting gap}
During my Master thesis, we rederived the selfconsistency equation for the superconducting gap from the definition of the Keldysh propagator.
One of the intermediate results we obtained was the following [Eq.~(3.16)]:
\begin{equation}
  \Delta = \frac{1}{8} N_F \lambda \!\!\int_{\;-\omega_c}^{\;+\omega_c} \!\!\d\epsilon\, \big[ f^K_{\up\dn}(\epsilon) - f^K_{\dn\up}(\epsilon) \big]
\end{equation}
This was the last equation we obtained before assuming an equilibrium distribution function, and therefore a natural startpoint for finding an appropriate nonequilibrium selfconsistency equation.
This equation may be rewritten in terms of the $2\times2$ propagator $\U{f}^K$:
\begin{equation}
  \Delta = \frac{1}{8} N_F \lambda \!\!\int_{\;-\omega_c}^{\;+\omega_c} \!\!\d\epsilon\, \tr\left[ (-i\S{2})\U{f}^K \right]
\end{equation}
This seems like a reasonable result, as we know from before that the parts of the anomalous propagator proportional to $i\S{2}$ is the singlet component.
Furthermore, we by definition have the identity $\U{f}^K(-\epsilon) = \TU{f}^{K*}(+\epsilon)$, which we can use to rewrite this as an integral over only positive energies:
\begin{equation}
  \Delta = \frac{1}{8} N_F \lambda \!\!\int_{0}^{\;\;\omega_c} \!\!\d\epsilon\, \tr\left[ (-i\S{2})\big(\U{f}{}^K + \TU{f}{}^{K*}\big)\right]
\end{equation}
In other words, if we know the $4\times4$ Keldysh propagator~$\U{G}^K$, then the selfconsistent gap can be calculated directly from its off-diagonal blocks:
\begin{equation}
  \Delta = \frac{1}{8} N_F \lambda \!\!\int_{0}^{\;\;\omega_c} \!\!\d\epsilon\, \tr\left[ (-i\S{2})\big(\U{G}{}^K_{12} + \U{G}{}^{K*}_{21}\big)\right]
\end{equation}
We could rewrite this using explicit Nambu-space traces, where $(\N{1} \pm i\N{2})/2$ can be used to extract the off-diagonal blocks of $\U{G}{}^K$.
However, in contrast to the other observables we looked at, the equation above only becomes more complicated if we try rewriting it in this way, since we then end up with separate equations for $\re(\Delta)$ and $\im(\Delta)$.
I therefore intend to use this form.


\clearpage
\section{Spin-orbit coupling}
To generalize to nanowires with spin-orbit coupling, we need to replace the derivatives $\nabla \rightarrow \nabla - i[\U{A}, \,\cdot\,]$ in the matrix current.
This results in some additional terms in both the supercurrent and resistive current:
\begin{align}
  \U{S} &\rightarrow \U{S} - i\U{S}' \\
  \U{R} &\rightarrow \U{R} - i\U{R}'
\end{align}
When the factor $-i$ is explicitly written in front of the gauge contributions $\U{S}'$ and $\U{R}'$, their definitions become just like $\U{S}$ and $\U{R}$, except that $\nabla \rightarrow [\U{A}, \,\cdot\,]$:
\begin{align}
  \U{S}' &= \U{G}^R [\U{A}, \U{G}^R] \U{H} - \U{H} \U{G}^A [\U{A}, \U{G}^A],  \\
  \U{R}' &= [\U{A}, \U{H}] - \U{G}^R [\U{A}, \U{H}] \U{G}^A.
\end{align}
The explicit structure of the spin-orbit field $\U{A}$ in Nambu-space is:
\begin{equation}
  \U{A} = 
  \begin{pmatrix}
    \U{a} & 0 \\
    0 & -\U{a}^*
  \end{pmatrix}
\end{equation}
From this, we see that we may write the identity:
\begin{equation}
  \U{A} = -\N{1}\U{A}^*\N{1}
\end{equation}
Combined with the previously derived identities for the transformations of $\U{G}^R$, $\U{G}^A$, $\U{H}$ when complex-conjugated and sandwiched between $\N{1}$-matrices:
\begin{align}
  \U{S}'(-\epsilon) &= +\N{1}\U{S}'{}^*(+\epsilon)\N{1} \\
  \U{R}'(-\epsilon) &= +\N{1}\U{S}'{}^*(+\epsilon)\N{1}
\end{align}
In other words, these have a sign change compared to the transformations of $\U{S}$ and $\U{R}$.
Denoting either gauge contribution $\U{S}'$ and $\U{R}'$ with the generic $\U{I}'$, the changes to the charge, spin, heat, and spin-heat currents become:
\begin{align}
  I_n' &= -\frac{i}{8}N_F D \int_{0}^{\infty} \d\epsilon \tr\big[ \N{3} \S{n} (\U{I}' + \N{1}\U{I}'{}^*\N{1}) \big]\\
  J_n' &= -\frac{i}{8}N_F D \int_{0}^{\infty} \d\epsilon \tr\big[ \epsilon \S{n} (\U{I}' - \N{1}\U{I}'{}^*\N{1}) \big]
\end{align}
Using the cyclic rule and that $\N{1}\N{3}\N{1} = -\N{3}$:
\begin{align}
  I_n' &= -\frac{i}{8}N_F D \int_{0}^{\infty} \d\epsilon \tr\big[ \N{3} \S{n} (\U{I}' - \U{I}'{}^*) \big]\\
  J_n' &= -\frac{i}{8}N_F D \int_{0}^{\infty} \d\epsilon \tr\big[ \epsilon \S{n} (\U{I}' - \U{I}'{}^*) \big]
\end{align}
Recognizing the parentheses as $2i\im(\U{I}')$:
\begin{align}
  I_n' &= +\frac{1}{4}N_F D \int_{0}^{\infty} \d\epsilon \im\tr\big[ \N{3} \S{n} \U{I}' \big]\\
  J_n' &= +\frac{1}{4}N_F D \int_{0}^{\infty} \d\epsilon \im\tr\big[ \epsilon \S{n} \U{I}' \big]
\end{align}
Now, in general $\im(u) = \re(-iu)$, so the above can be rewritten as:
\begin{align}
  I_n' &= \frac{1}{4}N_F D \int_{0}^{\infty} \d\epsilon \re\tr\big[ \N{3} \S{n} (-i\U{I}') \big]\\
  J_n' &= \frac{1}{4}N_F D \int_{0}^{\infty} \d\epsilon \re\tr\big[ \epsilon \S{n} (-i\U{I}') \big]
\end{align}
Now, $-i\U{I}'$ refers to either $-i\U{S}'$ or $-i\U{R}'$, which are precisely the terms that we added to the matrix currents $\U{S}$ and $\U{R}$ when we introduced the gauge-covariant derivatives above.
Thus, the conclusion is that even in the presence of spin-orbit coupling, the previously derived equations for the charge, spin, heat, and spin-heat currents hold, if we just straight-forwardly replace regular derivatives with covariant derivatives in the matrix current.





\clearpage
\section{Density of states}
Based on Eschrig's review from 2015 [Eqs.~(47) and (63)], it seems like a reasonable definition of the spin-resolved density of states is ($n=0,\ldots,3$):
\begin{align}
  N_n = \frac{1}{2}N_F \re\tr\big[\S{n} \U{g}^R\big]
\end{align}
In terms of these four coefficients, the density of stated for spins polarized along some direction $\B{\nu}$ would be:
\begin{align}
  N_{\B{\nu}}  = N_0 + \B{\nu} \cdot \B{N},
\end{align}
where $\B{N} = (N_1, N_2, N_3)$.
If we average over all $\B\nu$, we get back the usual definition of the spin-independent density of states~$N_0$.
On the other hand, if we let $\B\nu = \pm\B{e}_z$, this reproduces Eqs.~(2.39--40) in I.~Gomperud's thesis for $N_\up$ and $N_\dn$.
Thus, this definition seems consistent with previous results.

\section{Summary}
On the previous pages, I have rewritten all the nonequilibrium observables in terms of only the positive-energy propagators.
This has several benefits.
First of all, every equation except the superconducting gap ended up with being explicitly real quantities, which is a nice quality for an equation that is supposed to represent real physical observables.
Secondly, this means that we only need to solve the Usadel equation for positive energies as before, which saves computation time.
Finally, while the trace-tricks we derived during the project with Tom are nice for recasting the Usadel equation in a numerically suitable form, evaluating e.g. charge and spin currents is actually a bit easier to implement the ``old-fashioned way''.

Interestingly, the cancellations that happened when we rewrote the observables in this way, all lead to the same result: if we just integrate over positive energies, and take twice the real part of that result, we're done.







\clearpage
\section{Accumulation}
The presence of nonequilibrium potentials can lead to imbalances in the quasiparticle distributions, which manifest as observable charge, spin, and heat accumulations.
In this section, I will first derive how these imbalances in the quasiparticle distributions can be related to the propagators, and then relate this to the accumulations discussed above.

There are four relevant species of quasiparticles in the system, namely electrons and holes with two possible spin projections.
Their densities can be written in terms of the spin-resolved creation and annihilation operators,
\begin{align}
  N_{e\up} &\equiv \big\langle \psi_{\up}^\dagger \psi_{\up}^{\phantom{\dagger}} \big\rangle, \\
  N_{e\dn} &\equiv \big\langle \psi_{\dn}^\dagger \psi_{\dn}^{\phantom{\dagger}} \big\rangle, \\
  N_{h\up} &\equiv \big\langle \psi_{\up}^{\phantom{\dagger}} \psi_{\up}^\dagger \big\rangle, \\
  N_{h\dn} &\equiv \big\langle \psi_{\dn}^{\phantom{\dagger}} \psi_{\dn}^\dagger \big\rangle.
\end{align}
If we use the notation $\psi_e = \psi$ and $\psi_h = \psi^\dagger$ for the electron and hole creation operators, respectively, we can summarize the above as a single equation,
\begin{align}
  N_{\tau\sigma} &\equiv \big\langle \psi_{\tau\sigma}^\dagger \psi_{\tau\sigma}^{\phantom{\dagger}} \big\rangle.
\end{align}
The creation and annihilation operators should in practice be evaluated at the same position~$\bm{r}$ and time~$t$.
However, to make sure the derivation is rigorous, I will keep two positions and times during the derivation:
\begin{align}
  N_{\tau\sigma}(\bm{r}',t'; \bm{r},t) &\equiv \big\langle \psi_{\tau\sigma}^\dagger(\bm{r}',t')\, \psi_{\tau\sigma}^{\phantom{\dagger}}(\bm{r},t) \big\rangle.
\end{align}

Let us now express the above in terms of non-quasiclassical propagators.
We can write $2AB = \{A, B\} + [A, B]$ for arbitrary operators $A$ and $B$:
\begin{equation}
\begin{aligned}
  2N_{\tau\sigma}
  &=   \big\langle \big\{ \psi_{\tau\sigma}^\dagger(\bm{r}',t'),\, \psi_{\tau\sigma}^{\phantom{\dagger}}(\bm{r},t) \big\} \big\rangle \\
  &\,+ \big\langle \big[  \psi_{\tau\sigma}^\dagger(\bm{r}',t'),\, \psi_{\tau\sigma}^{\phantom{\dagger}}(\bm{r},t) \big]  \big\rangle. 
\end{aligned}
\end{equation}
We also know that the step function satisfies $\theta(t-t') + \theta(t'-t) = 1$:
\begin{equation}
\begin{aligned}
  2N_{\tau\sigma}
  &=   \big\langle \big\{ \psi_{\tau\sigma}^\dagger(\bm{r}',t'),\, \psi_{\tau\sigma}^{\phantom{\dagger}}(\bm{r},t) \big\} \big\rangle \,\theta(t-t') \\
  &\,+ \big\langle \big\{ \psi_{\tau\sigma}^\dagger(\bm{r}',t'),\, \psi_{\tau\sigma}^{\phantom{\dagger}}(\bm{r},t) \big\} \big\rangle \,\theta(t'-t) \\
  &\,+ \big\langle \big[  \psi_{\tau\sigma}^\dagger(\bm{r}',t'),\, \psi_{\tau\sigma}^{\phantom{\dagger}}(\bm{r},t) \big]  \big\rangle. 
\end{aligned}
\end{equation}
If we now consult the definitions of the non-quasiclassical propagators [Eq.~(2.12--2.14) in my Master thesis], we see that the above can be written:
\begin{equation}
\begin{aligned}
  2N_{\tau\sigma}
   =& \pm i G^R_{\tau\tau\sigma\sigma}(\bm{r}, t; \bm{r}', t') \\
    & \mp i G^A_{\tau\tau\sigma\sigma}(\bm{r}, t; \bm{r}', t') \\
    & \mp i G^K_{\tau\tau\sigma\sigma}(\bm{r}, t; \bm{r}', t') .
\end{aligned}
\end{equation}
The top signs correspond to $\tau = e$ and the bottom ones to $\tau = h$, and this follows from our definitions of the propagator matrices.%
\footnote{According to Eq.~(2.14) in my Master thesis, the Keldysh propagator for electrons is:
  \begin{equation} \quad G^K_{ee} = +i\big\langle \big[\psi_e^{\dagger}(\bm{r}',t'),\,\psi_e^{\phantom\dagger}(\bm{r},t) \big] \big\rangle. \end{equation}
  However, Eq.~(2.20) in my thesis shows that $G^K_{hh} \equiv -G^{K\ast}_{ee}$, which implies that:
  \begin{equation} \quad G^K_{hh} = -i\big\langle \big[\psi_e^{\phantom\dagger}(\bm{r}',t'),\,\psi_e^{\dagger}(\bm{r},t) \big] \big\rangle = -i\big\langle \big[\psi_h^{\dagger}(\bm{r}',t'),\,\psi_h^{\phantom\dagger}(\bm{r},t) \big] \big\rangle. \end{equation}
  This is why we get different signs for the $G^K$ contributions for electrons and holes. 
  A similar analysis can be done for $G^R$ and $G^A$, although these are less relevant here.\\[1ex]}
The retarded propagator~$G^R$ and advanced propagator~$G^A$ only depend on the equilibrium state of the system, and can be discarded since we are looking for the \emph{nonequilibrium quasiparticle accumulations}.\footnote{Rammer and Smith makes a similar argument below Eq.~(2.86) in [RMP 58, 323 (1986)].}
Explicitly, we therefore conclude:\\[-3ex]
\begin{equation}
\begin{aligned}
  N_{e\up} &= -\frac{i}{2} G^K_{11}(\bm{r}, t; \bm{r}', t'), \\
  N_{e\dn} &= -\frac{i}{2} G^K_{22}(\bm{r}, t; \bm{r}', t'), \\
  N_{h\up} &= +\frac{i}{2} G^K_{33}(\bm{r}, t; \bm{r}', t'), \\
  N_{h\dn} &= +\frac{i}{2} G^K_{44}(\bm{r}, t; \bm{r}', t'). \\
\end{aligned}
\end{equation}
These equations give us all nonequilibrium quasiparticle accumulations in terms of the non-quasiclassical propagators.

Let us now consider the quasiclassical limit, and let $(\bm{r}',t') \rightarrow (\bm{r},t)$.
In the same way as we obtained Eq.~(3.13) in my Master thesis, we can derive that:
\begin{align}
  \underline{G}^K(\bm{r},t; \bm{r},t) \approx -\frac{i}{2} N_0 \int \d\epsilon \, \underline{g}^K(\bm{r}, t, \epsilon).
\end{align}
Substituted into the equations above, we find that the quasiclassical expressions for the quasiparticle accumulations $N_{\tau\sigma}(\bm{r},t)$ can be written:
\begin{equation}
\begin{aligned}
  N_{e\up}(\bm{r}, t) &= -\frac{1}{4} N_0 \int \d\epsilon \, g^K_{11}(\epsilon, \bm{r}, t), \\
  N_{e\dn}(\bm{r}, t) &= -\frac{1}{4} N_0 \int \d\epsilon \, g^K_{22}(\epsilon, \bm{r}, t), \\
  N_{h\up}(\bm{r}, t) &= +\frac{1}{4} N_0 \int \d\epsilon \, g^K_{33}(\epsilon, \bm{r}, t), \\
  N_{h\dn}(\bm{r}, t) &= +\frac{1}{4} N_0 \int \d\epsilon \, g^K_{44}(\epsilon, \bm{r}, t). \\
\end{aligned}
\end{equation}

Let us now consider the physical consequences of such quasiparticle accumulations.
Each electron contributes a charge $+e$, and each hole a charge $-e$, where we use $e<0$ as usual.
Thus, the charge imbalance~$\rho_e$ is:
\begin{equation}
  \rho_e \equiv e(N_{e\up} + N_{e\dn} - N_{h\up} - N_{h\dn}).
\end{equation}
Similarly, each spin-up particle contributes a spin $+\hbar/2$, and each spin-down particle a spin $-\hbar/2$.
For their antiparticles, the spin signs are reversed.\footnote{The creation operator for a spin-up hole~$\psi_{h\up}^\dagger$ is the same as the annihilation operator for a spin-up electron~$\psi_{e\up}$. This means that annihilating a spin-up electron should be physically equivalent to creating a spin-up hole. We know that annihilating a spin-up electron gives a change $-\hbar/2$ to the system spin, which is the same as we get by creating a spin-down electron. Thus, spin-up holes and spin-down electrons have the same spin.}
We can therefore define the spin accumulation as:
\begin{equation}
  \rho_\sigma \equiv \frac{\hbar}{2} (N_{e\up} - N_{e\dn} - N_{h\up} + N_{h\dn}).
\end{equation}
%Combining the results above with the equations we derived for $N_{\tau\sigma}$:
%\begin{align}
%  \rho_e        &= -\frac{1}{4} \,e\,           N_0 \int \d\epsilon \, (g^K_{11} + g^K_{22} + g^K_{33} + g^K_{44}), \\
%  \rho_\sigma   &= -\frac{1}{4} \frac{\hbar}{2} N_0 \int \d\epsilon \, (g^K_{11} - g^K_{22} + g^K_{33} - g^K_{44}).
%\end{align}
%Rewriting the above in terms of traces:
%\begin{align}
%  \rho_e        &= -\frac{1}{16} \,e\,           N_0 \int \d\epsilon \, \text{Tr}[\hat{\tau}_0 \underline{\sigma}_0 \hat{\underline{G}}], \\
%  \rho_\sigma   &= -\frac{1}{16} \frac{\hbar}{2} N_0 \int \d\epsilon \, \text{Tr}[\hat{\tau}_0 \underline{\sigma}_3 \hat{\underline{G}}].
%\end{align}
%This is a quasiclassical result for the charge and spin accumulation.
Let us now consider the heat accumulation~$\kappa_e$.
The relevant metric is the quasiparticle energy~$\epsilon = E - \mu$ relative to the Fermi level~$\mu$, where $\epsilon > 0$ for electrons and $\epsilon < 0$ for holes.
Neither charge nor spin matters here. Thus:
\begin{equation}
  \kappa_e \equiv \epsilon(N_{e\up} + N_{e\dn} + N_{h\up} + N_{h\dn}).
\end{equation}
Finally, one can make another independent linear combination of the above quasiparticle densities, yielding the so-called spin-heat accumulation:
\begin{equation}
  \kappa_\sigma \equiv \epsilon(N_{e\up} - N_{e\dn} + N_{h\up} - N_{h\dn}).
\end{equation}
Formally, the prefactors~$\epsilon$ above should be added after switching to the mixed representation ($\bm{r}\pm\bm{r}'$ and $t \pm t'$) and Fourier transforming, but before performing the quasiclassical approximation itself.
If we now combining all the results above with the equations we derived for $N_{\tau\sigma}$:
\begin{align}
  \rho_e        &= -\frac{1}{4} N_0 \int \d\epsilon \, \,e\,           (g^K_{11} + g^K_{22} + g^K_{33} + g^K_{44}), \\
  \rho_\sigma   &= -\frac{1}{4} N_0 \int \d\epsilon \, \frac{\hbar}{2} (g^K_{11} - g^K_{22} + g^K_{33} - g^K_{44}), \\
  \kappa_e      &= -\frac{1}{4} N_0 \int \d\epsilon \, \,\epsilon\,    (g^K_{11} + g^K_{22} - g^K_{33} - g^K_{44}), \\
  \kappa_\sigma &= -\frac{1}{4} N_0 \int \d\epsilon \, \,\epsilon\,    (g^K_{11} - g^K_{22} - g^K_{33} + g^K_{44}).
\end{align}
Rewriting the above in terms of traces, we finally obtain:
\begin{align}
  \rho_e        &= -\frac{1}{16} N_0 \int \d\epsilon \,  \,e\,           \text{Tr}[\hat{\tau}_0 \underline{\sigma}_0 \hat{\underline{g}}^K], \\
  \rho_\sigma   &= -\frac{1}{16} N_0 \int \d\epsilon \,  \frac{\hbar}{2} \text{Tr}[\hat{\tau}_0 \underline{\sigma}_3 \hat{\underline{g}}^K], \\
  \kappa_e      &= -\frac{1}{16} N_0 \int \d\epsilon \,  \,\epsilon\,    \text{Tr}[\hat{\tau}_3 \underline{\sigma}_0 \hat{\underline{g}}^K], \\
  \kappa_\sigma &= -\frac{1}{16} N_0 \int \d\epsilon \,  \,\epsilon\,    \text{Tr}[\hat{\tau}_3 \underline{\sigma}_3 \hat{\underline{g}}^K].
\end{align}
These are in fact the same expressions that we have been using previously, but we now have a derivation from the definitions of the propagators.
